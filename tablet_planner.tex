\documentclass[landscape]{article}
\usepackage[paperwidth=254mm,paperheight=452mm,margin=0mm]{geometry}
\usepackage{tikz}
\usetikzlibrary{calc}
\usepackage{fontspec}
\usepackage{xeCJK}
\usepackage{fancyhdr}
\usepackage{datetime2}
\usepackage{amssymb}

% フォント設定(システムにある日本語フォントを使用)
\setmainfont{DejaVu Sans}
\setCJKmainfont{Noto Sans CJK JP}
\setCJKsansfont{Noto Sans CJK JP}

% ヘッダー・フッター設定
\pagestyle{fancy}
\fancyhf{}
\renewcommand{\headrulewidth}{0pt}
\renewcommand{\footrulewidth}{0pt}

% 色設定
\definecolor{lightgray}{RGB}{240,240,240}
\definecolor{medgray}{RGB}{200,200,200}
\definecolor{darkgray}{RGB}{100,100,100}
\definecolor{accent}{RGB}{100,150,200}

\setlength{\parindent}{0pt}
\setlength{\parskip}{0pt}

\begin{document}

% ページ1: 週間スケジュール
\begin{tikzpicture}[remember picture,overlay]
\node[anchor=north west] at (current page.north west) {
\begin{minipage}[t][452mm][t]{254mm}

% ヘッダー
\begin{tikzpicture}
\fill[accent!30] (0,0) rectangle (254mm,32mm);
\node[anchor=west,font=\fontsize{36}{40}\selectfont\bfseries] at (10mm,16mm) {週間スケジュール};
\node[anchor=east,font=\fontsize{20}{24}\selectfont] at (244mm,16mm) {2025年 \_\_月 \_\_週};
\end{tikzpicture}

\vspace{5mm}

% 時間軸付き週間表
\begin{tikzpicture}[x=1mm,y=1mm]
% 曜日ヘッダー
\foreach \x/\day in {0/月,1/火,2/水,3/木,4/金,5/土,6/日} {
  \pgfmathsetmacro{\xpos}{10+\x*34.57}
  \pgfmathsetmacro{\xposend}{10+(\x+1)*34.57}
  \fill[lightgray] (\xpos,380) rectangle (\xposend,396);
  \draw[thick] (\xpos,380) rectangle (\xposend,396);
  \pgfmathsetmacro{\xcenter}{\xpos+17.285}
  \node[font=\fontsize{18}{22}\selectfont\bfseries] at (\xcenter,388) {\day};
}

% 時間軸と罫線
\foreach \h in {6,7,...,23} {
  \pgfmathsetmacro{\y}{380-(\h-6)*21.1}
  \draw[medgray] (10,\y) -- (252,\y);
  \node[anchor=east,font=\fontsize{12}{14}\selectfont] at (9,\y) {\h:00};
}

% 縦線(曜日の区切り)
\foreach \x in {0,1,...,7} {
  \pgfmathsetmacro{\xpos}{10+\x*34.57}
  \draw[thick] (\xpos,0) -- (\xpos,396);
}

% 外枠
\draw[very thick] (10,0) rectangle (252,396);
\end{tikzpicture}

\end{minipage}
};
\end{tikzpicture}

\newpage

% ページ2: デイリープランナー
\begin{tikzpicture}[remember picture,overlay]
\node[anchor=north west] at (current page.north west) {
\begin{minipage}[t][452mm][t]{254mm}

% ヘッダー
\begin{tikzpicture}
\fill[accent!30] (0,0) rectangle (254mm,32mm);
\node[anchor=west,font=\fontsize{36}{40}\selectfont\bfseries] at (10mm,16mm) {デイリープランナー};
\node[anchor=east,font=\fontsize{20}{24}\selectfont] at (244mm,16mm) {\_\_月\_\_日(\_\_)};
\end{tikzpicture}

\vspace{5mm}

% 左側:時間割
\begin{tikzpicture}[x=1mm,y=1mm]
% 時間軸
\draw[very thick] (10,20) rectangle (127,372);
\node[anchor=south,font=\fontsize{16}{20}\selectfont\bfseries] at (68.5,374) {スケジュール};

\foreach \h in {6,7,...,23} {
  \pgfmathsetmacro{\y}{372-(\h-6)*19.56}
  \pgfmathsetmacro{\yend}{\y+19.56}
  \pgfmathsetmacro{\ycenter}{\y+9.78}
  \draw[medgray] (10,\y) -- (127,\y);
  \node[anchor=east,font=\fontsize{14}{16}\selectfont] at (23,\ycenter) {\h:00};
  \draw[thick] (25,\y) -- (25,\yend);
}

% 右側:TODO & メモ
\draw[very thick] (132,236) rectangle (244,374);
\node[anchor=south,font=\fontsize{16}{20}\selectfont\bfseries] at (188,376) {TODO};

\foreach \i in {1,2,...,12} {
  \pgfmathsetmacro{\y}{368-\i*11}
  \pgfmathsetmacro{\ycenter}{\y+5.5}
  \draw[lightgray] (137,\y) -- (239,\y);
  \node[anchor=west,font=\fontsize{12}{14}\selectfont] at (139,\ycenter) {$\square$};
}

% メモエリア
\draw[very thick] (132,20) rectangle (244,228);
\node[anchor=south,font=\fontsize{16}{20}\selectfont\bfseries] at (188,230) {メモ};

\foreach \i in {1,2,...,18} {
  \pgfmathsetmacro{\y}{220-\i*11}
  \draw[lightgray] (137,\y) -- (239,\y);
}
\end{tikzpicture}

\end{minipage}
};
\end{tikzpicture}

\newpage

% ページ3: マンスリーカレンダー
\begin{tikzpicture}[remember picture,overlay]
\node[anchor=north west] at (current page.north west) {
\begin{minipage}[t][452mm][t]{254mm}

% ヘッダー
\begin{tikzpicture}
\fill[accent!30] (0,0) rectangle (254mm,32mm);
\node[anchor=west,font=\fontsize{36}{40}\selectfont\bfseries] at (10mm,16mm) {マンスリーカレンダー};
\node[anchor=east,font=\fontsize{20}{24}\selectfont] at (244mm,16mm) {2025年 \_\_月};
\end{tikzpicture}

\vspace{5mm}

% カレンダーグリッド
\begin{tikzpicture}[x=1mm,y=1mm]
% 曜日ヘッダー
\foreach \x/\day in {0/日,1/月,2/火,3/水,4/木,5/金,6/土} {
  \pgfmathsetmacro{\xpos}{10+\x*34.57}
  \pgfmathsetmacro{\xposend}{10+(\x+1)*34.57}
  \pgfmathsetmacro{\xcenter}{\xpos+17.285}
  \fill[lightgray] (\xpos,377) rectangle (\xposend,398);
  \draw[thick] (\xpos,377) rectangle (\xposend,398);
  \node[font=\fontsize{16}{20}\selectfont\bfseries] at (\xcenter,387.5) {\day};
}

% 6週分のグリッド
\foreach \row in {0,1,...,5} {
  \foreach \col in {0,1,...,6} {
    \pgfmathsetmacro{\y}{377-(\row+1)*58.67}
    \pgfmathsetmacro{\yend}{\y+58.67}
    \pgfmathsetmacro{\xpos}{10+\col*34.57}
    \pgfmathsetmacro{\xposend}{10+(\col+1)*34.57}
    \draw[thick] (\xpos,\y) rectangle (\xposend,\yend);
  }
}
\end{tikzpicture}

\end{minipage}
};
\end{tikzpicture}

\end{document}